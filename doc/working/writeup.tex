% Created 2024-06-13 Thu 12:57
% Intended LaTeX compiler: pdflatex
\documentclass[letterpaper, 11pt]{article}
                      \usepackage{lmodern} % Ensures we have the right font
\usepackage[T1]{fontenc}
\usepackage[utf8]{inputenc}
\usepackage{graphicx}
\usepackage{amsmath, amsthm, amssymb}
\usepackage[table, xcdraw]{xcolor}
\definecolor{bblue}{HTML}{0645AD}
\usepackage[colorlinks]{hyperref}
\hypersetup{colorlinks, linkcolor=blue, urlcolor=bblue}
\usepackage{titling}
\setlength{\droptitle}{-6em}
\setlength{\parindent}{0pt}
\setlength{\parskip}{1em}
\usepackage[stretch=10]{microtype}
\usepackage{hyphenat}
\usepackage{ragged2e}
\usepackage{subfig} % Subfigures (not needed in Org I think)
\usepackage{hyperref} % Links
\usepackage{listings} % Code highlighting
\usepackage[top=1in, bottom=1.25in, left=1.55in, right=1.55in]{geometry}
\renewcommand{\baselinestretch}{1.15}
\usepackage[explicit]{titlesec}
\pretitle{\begin{center}\fontsize{20pt}{20pt}\selectfont}
\posttitle{\par\end{center}}
\preauthor{\begin{center}\vspace{-6bp}\fontsize{14pt}{14pt}\selectfont}
\postauthor{\par\end{center}\vspace{-25bp}}
\predate{\begin{center}\fontsize{12pt}{12pt}\selectfont}
\postdate{\par\end{center}\vspace{0em}}
\titlespacing\section{0pt}{5pt}{5pt} % left margin, space before section header, space after section header
\titlespacing\subsection{0pt}{5pt}{-2pt} % left margin, space before subsection header, space after subsection header
\titlespacing\subsubsection{0pt}{5pt}{-2pt} % left margin, space before subsection header, space after subsection header
\usepackage{enumitem}
\setlist{itemsep=-2pt} % or \setlist{noitemsep} to leave space around whole list
\usepackage{tabularx}
\author{Stefan Harris}
\date{}
\title{Computer Science H446 Coursework Project: Pac-Man Game\\\medskip
\large Candidate Number: 1419 Centre Number: 19268}
\hypersetup{
 pdfauthor={Stefan Harris},
 pdftitle={Computer Science H446 Coursework Project: Pac-Man Game},
 pdfkeywords={},
 pdfsubject={},
 pdfcreator={Emacs 29.3 (Org mode 9.6.24)}, 
 pdflang={English}}
\usepackage{biblatex}
\addbibresource{/home/stefan/documents/programming/pac-man-clone/doc/working/bib/sources.bib}
\begin{document}

\maketitle
\tableofcontents


\section{Introduction}
\label{sec:orgafdfc6a}
This project aims to develop a program serving as a recreation of the 1980 coin-op arcade game Pac-Man, using both the household high-level programming language Python and a 3rd party library to enable game development, Pygame.
The project aims at creating a rather conservative recreation of the original.
The core gameplay loop will be retained as much as possible by researching the original game’s mechanics, namely the mechanics of the ghosts: the algorithms that dictate where they will move, where they will move to and how to react to key game play moments.
Any features outside gameplay will be mostly cosmetic and or serves as a quality-of-life improvement for the player.
\section{Analysis}
\label{sec:orgb86c8af}
\subsection{Identifying the Problem (An Explanation of Pac-Man)}
\label{sec:org230626c}
Pac-Man is a 1980 arcade game developed and published by Bandai Namco in Japan.
And published by Midway Games in North America.
Despite initially receiving lukewarm reception in Japan, the game showed both critical and commercial success outside the nation and the franchise remains one of the highest-grossing video game series of all time.

Development of Pac-man begun in 1979, directed by Toru Iwatani with a nine-man team.
Iwatani noticed that most arcade games of the time appealed more to a masculine audience, with conflict being quite unanimous.
Discontent with this, he aimed to create a game that would appeal to woman as well as men.
A colourful art style and game play loop that would appeal to all ages was the idea.

The core gameplay revolves around Pac-Man and his goal, to eat all the pellets/dots in the maze without being caught by the pursuing ghosts.
Pac-man can turn the tables by tactically eating the large dots known as power pellets.
This will frighten the ghosts, making them vulnerable to Pac-Man eating them for bonus points.
Difficulty increases the more levels the player completes, with the ghosts becoming more aggressive among other subtle changes, like power pellets no longer fully functioning.

To add variance to the gameplay, each ghost has their own unique behaviour to catch Pac-Man, giving the illusion that they are working together and making them seem smarter than they are.
They also alternate between behaviour states during a level, namely chase mode –when they are pursuing Pac-Man, and scatter mode – when they flee to the maze’s corners to give the player a breather.

\begin{figure}[htbp]
\centering
\includegraphics[width=0.5\linewidth]{./img/analysis/pacman-start.jpg}
\caption{\label{fig:pacman-start}The beginning of a game of Pac-Man.}
\end{figure}

\subsection{Solving the Problem using Computational Methods}
\label{sec:orgf19ad91}
\label{Solving the Problem using Computational Methods}
One of the reasons I chose Pac-Man for my coursework is because I believe it serves as a good example on how computational methods can be applied to a large project to make it more approachable.

\subsubsection{Abstraction}
\label{sec:orgc4bcbbe}
Abstraction helps in reducing the amount of hardware resources taken up by the program, ensuring that performance is optimal on the range of target systems.
An abstract model will serve as a representation of reality, only functioning in the ways it needs to.

Pac-Man is already an abstracted form of a chasing simulation.
The game takes place on a 2D plane, yet still provides a good representation of a real-life chase scenario.
The omission of a 3rd dimension results in far less lines of code, storage, and processing, a necessity when acknowledging the limited hardware of the time.
All the entities are sprites consisting of only a couple colours and frames of animation, a drastic simplification of how the characters are depicted in the game’s promotional material and arcade cabinet.
The maze does not represent any real-life location and is represented by simple outlines for walls.
The entities can only move in cardinal directions.

\begin{figure}[htbp]
\centering
\includegraphics[width=0.33\linewidth]{./img/analysis/arcade-cabinet.jpg}
\caption{\label{fig:arcade-cabinet}An original arcade cabinet. Note how the art on the side doesn't match the game's graphics, but more-so its bright colours.}
\end{figure}

With the source material already being abstracted, it is easy to find and implement the core mechanics into my project such as the player movement, ghost behaviour and ghost states while ignoring more superfluous aspects like cut-scenes between levels.

\subsubsection{Decomposition}
\label{sec:org76c1638}
For complex problems like this, decomposition is invaluable for breaking it down into simpler, more manageable chunks that can be prioritised, developed, and tested individually.
It makes life so much easier on the developer’s end.

This solution is no exception, it can be decomposed into its core components.

For example, the core gameplay loop of Pac-Man can be decomposed into a series of entities interacting with each other: Pac-Man, pellets, ghosts, bonus fruit etc.
Each entity can then be further decomposed into their core behaviours.
So, for Pac-man that will be his movement, his eating and dying.
For ghosts, it is their movement algorithm and behaviour states.
Once the game is fully decomposed, the components can then be prioritised, such as the maze being implemented before the gameplay since the maze will house all the gameplay.

\subsubsection{Logic, Branching and Iteration}
\label{sec:org4084bcc}
These are pivotal for adding interactivity and dynamism to the solution.
Without these the code is completely linear and with little to no meaningful interactivity, which is not optimal for a game, a program that is supposed to react to user input.

Iteration is essential because it allows the game to repeat operations indefinitely, significantly reducing the number of lines in the code base.
It also allows the game to run indefinitely via a game loop.

Logic is an integral part of the gameplay; the ghosts will use logical comparisons to decide what will be the next best direction to get themselves closer to Pac-Man or to decide if a direction is eligible to move in or to decide if it is time to switch to a different behaviour state.

Branching in the form of if/else statements must be used for the game to respond to user input and produce an action.
The game can check for user inputs each frame using a series of if statements.
For example, when the player presses the left movement key, Pac-Man will move left, otherwise he will not.

Iteration in the form of for/while loops is integral for the game to run in real-time.
A main game loop will be used to clear the screen, update the entities, and refresh the display for each frame.
Iteration will also be used in entity behaviour; the ghosts will iterate through each cardinal direction to decide what direction to move in.

\subsubsection{Pipelining and Concurrent Thinking}
\label{sec:org9d99adc}
Pipelining allows for one process to start while another is finishing.
Concurrency is when multiple processes are split into time slices and ordered into a schedule to be worked on.

These methods allow for an increased program throughput, which is important for a game like this.
The game should run at a consistent frame rate, preferably 60 frames per second; all computation between frames needs to be done within a \textasciitilde{}16.67ms window.
Major drops in framerate will mean choppy movement, stuttering and less responsive controls, the latter being the most unbearable since the game can demand quick reactions.
The throughput pipelining and concurrency offers will ensure that all computation between frames will be done within the window, keeping performance at an optimum.

Concurrent processing via scheduling algorithms is present in almost every modern operating system.
The same can be said for pipelining, which is a mainstay in most contemporary CPU architectures.
These two methods will not need to be explicitly implemented in my code because these are implicit optimizations that are autonomous.

\subsubsection{Performance Modelling}
\label{sec:org61481b4}
Performance modelling means analysing the performance of algorithms to determine what processes are taking up the most time and addressing them accordingly.
It is vital for the game to run smoothly so that there is minimal delay between the player’s input and what appears on screen.
Excessive frame drops and delayed input can make the game cumbersome to play.
Knowledge of the scalability of algorithms will help in predicting any future performance issues and optimisations.
Big  notation is suitable for this task.
For example, if I know that an algorithm has a time complexity of , I can seek an algorithm that scales better, stopping the poor scalability form becoming an issue later in development.

Python’s built-in libraries such as ‘time’ will be used to measure the elapsed time between an algorithm starting and ending, with this information, it is possible to find the percentage of time an algorithm takes in a single frame or even the approximate number clock cycles it uses.
Optimisations can then be achieved based on this data.
However, knowing the Big O is arguably more beneficial as the time for an algorithm to compute will very across hardware configurations.

\subsubsection{Thinking Ahead}
\label{sec:orga5eee48}
This refers to the careful planning to make sure the project is completed to a good standard and within time.
Thinking ahead is useful as it allows you to determine how much time should be spent on each key part of the solution.

I can achieve this by dictating a programming paradigm that the solution will revolve around, which will be Object Oriented Programming (OOP).
By sticking to this, I can plan out how each entity of the game will translate to objects using Unified Modelling Language (UML) diagrams.
OOP lends itself well to reusability and modularity through inheritance and polymorphism.
Which is a big plus for saving time and for keeping the code base clean and simple.

Thinking ahead also refers to caching, the process where frequently accessed data can be temporarily stored somewhere for quick access when needed.
In software this can be a web cache in local storage, in hardware this can be a small primary storage inside a processor.
All modern consumer grade CPUs have a cache that is used implicitly, so I do not need to implement caching in my code.

\subsubsection{Stakeholders}
\label{sec:org06336fe}
This game can attract a variety of stakeholders, like how the original arcade release aimed to satisfy a wide demographic.
These stakeholders include:

\begin{itemize}
\item \textbf{Children} – Aged 5 to 12, male and female.
Children that have an interest in video games and or play video games as a pastime.
These children may not be very versed in technology but will be able to pick the game up easily due to the simple controls and easy to grasp winning and losing conditions.
They may also be attracted to the colourful appearance of Pac-Man and the ghosts.
The game was designed for an arcade setting, allowing it to be played in short bursts.
Suitable for children with short attention spans.
The score can serve as a gateway to competition amongst a group of children.
While the controls and graphics may be the most important to them, they may ask for a high score/leader board system to further facilitate competition.
While possible, implementing this is not on top of my priority list.
Implementing the gameplay first will not only satisfy all stakeholders, but also guarantees that the game will serve as a strong showcase of \hyperref[Solving the Problem using Computational Methods]{solving a problem using computational methods} and algorithmic thinking within the time that I have.
If I have enough time, I can add these extra features.

\item \textbf{Adults} – Aged 20 to 50, male and female.
 Adults that remember playing the original arcade realise during its peak or remember playing later versions or realises.
 They may be a fan of Pac-Man, looking for a simple way of playing a recreation of Pac-Man without having to toy with emulation or other methods.
 Akin to the children, some adults may not be versed with technology, so they may seek a simple way of playing Pac-Man, which this solution will serve.
 The simple controls and arcade-centric design allows adults to use the game to get some entertainment during their free time.
 While the simplicity of the game can be appealing for young children, it can become boring for adults, they may ask for variety \hyperref[Ms. Pac-Man]{seen in other releases, like multiple maze layouts}.
Again, I want to focus on implementing the core gameplay, with extra features being added if I have enough time.

\item \textbf{Programmers} – Likely hobbyists that have an interest in how retro games work.
These people are likely to have less of an interest in the game and more so on its code.
They may find use in an implementation of Pac-Man’s underlying algorithms in a modern high-level language, they can fork the code for use in their own projects, without having to do research or use an emulator’s disassembler and debugger to try and make sense of the algorithms.
These stakeholders may ask for the source code to be easy to read and understand.
An aspect of my solution that will be a priority.
\end{itemize}

\section{Researching Existing Solutions}
\label{sec:orgfd25db1}
To replicate the game play of the original Pac-Man, I needed to do some research on existing solutions.
This involved a dive into the original algorithm that dictates how the ghosts traverse the maze, their unique behaviour, and their various states.
With this knowledge I will be able to confidently translate these algorithms into Python and Pygame.

The commercial success of the original of Pac-Man naturally led to many other realises over the years.
Some offer simple quality-of-life improvements, to adding whole new mechanics.
I will assess these features and decide whether to implement them or not.

\subsection{Pac-Man (A Look into Ghost Movement and Behaviour)}
\label{sec:orgb34dbee}
One of the reasons I decided to do this project was because of this comprehensive video I watched a while ago from a YouTube channel called Retro Game Mechanics Explained.
The video goes over every facet of the ghosts’ behaviours in detail.

\begin{figure}[htbp]
\centering
\includegraphics[width=0.33\linewidth]{./img/analysis/retro-game-mech/ghost-states.jpg}
\caption{\label{fig:ghost-states}Video graphic showing the four states and how they interact \autocite{RetroGameMechanicsExplained}.}
\end{figure}

The video begins with an explanation of the four states and how each ghost moves between them.
There are 4 states: chase, scatter, frightened and eaten.
The ghosts will alternate between chase and scatter depending on a timer.
When Pac-Man eats a power pellet, they will temporarily enter frightened mode and then enter eaten mode when Pac-Man touches them.
Upon entering the ghost house, they will regenerate and either enter scatter or chase depending on the timer.
This is something that I will definitely add.

The video then explains the timings for scatter and chase mode, with the quirk of Blinky constantly being in chase mode when there are only a few dots left in the maze, something that I will also add.

An explanation of the movement algorithm is then given.
The ghosts use a targeting system to determine the tile to move towards.
Each cardinal direction is checked to see if it’s a valid direction to move in.
A direction is considered invalid if it leads into a wall or causes them move back the way they came.
Out of the valid directions, the distance between the tile the direction will lead to and the target is calculated.
The direction that yields the smallest number will be the direction the ghost will move in.
If all directions yield the same distance a priority system is used to pick one.
Up is given the highest priority, then left, down and right.
All of this will be included.

\begin{figure}[htbp]
\centering
\includegraphics[width=0.50\linewidth]{./img/analysis/retro-game-mech/choose-direction.jpg}
\caption{\label{fig:choose-direction}Video graphic showing Blinky choosing a direction \autocite{RetroGameMechanicsExplained}.}
\end{figure}

The various modes are then explained.
During scatter mode, the ghosts’ target tile is set to a corner of the maze.
When chasing, Blinky’s target tile is directly on Pac-Man, Pinky’s is four tiles ahead, Inky’s is dependant on Blinky’s and Pac-Mans postion, and Clyde is on Pac-Man, until he gets within eight tiles of him, in which case he will scatter away for several seconds.
When frightened the ghosts will u-turn and begin to move in random, eligible directions according to a random number generator (RNG).
All this will be added.

The video addresses a bug that occurs when multiplying the ‘up’ unit vector.
The code that handles applying magnitude to unit vectors treats the x and y components as one 16-bit value rather than two 8-bit ones, leading to the multiplication of the y component overflowing into the x.
For Pinky this means that her target tile is three tiles up and four left when Pac-man is facing upwards.
This is obviously a bug, therefore I won’t replicate it in my solution.

\begin{figure}[htbp]
\centering
\includegraphics[width=0.50\linewidth]{./img/analysis/retro-game-mech/pinky-bug.jpg}
\caption{\label{fig:pinky-bug}Video Graphic showing the overflow error. Multiplying \texttt{\$FF} by 4 makes \texttt{\$3FC}. The left most nibble overflows into the x component. \autocite{RetroGameMechanicsExplained}.}
\end{figure}

\subsection{Ms. Pac-Man}
\label{sec:orgfed377a}
\label{Ms. Pac-Man}
Released in 1982 by publisher Midway, Ms Pac-Man is the questionably licensed sequel to Namco’s original arcade release.

\begin{figure}[htbp]
\centering
\includegraphics[width=0.33\linewidth]{./img/analysis/ms-pac-man/start.jpg}
\caption{\label{fig:ms-pacman-start}Start of a game of Ms. Pac-Man. Note how the walls are now solid pink and that they’re now two sets of warp tunnels.}
\end{figure}

The gameplay does not stray far away from the original, instead it adds new quality-of-life improvements and new content that still retains the timelessness of the original’s gameplay:

\begin{itemize}
\item The maze walls are now filled with a solid colour that contrasts better with the black rather than being simple outlines that can blend into the background, especially so considering the blur CRT displays have, which were in ubiquitous during the era Pac-Man came out in.
This offers a nice usability feature for the visually impaired, who may struggle to distinguish the walls from the background.
This is something that I will certainly implement into my solution.
\item One gameplay enhancement comes from the ghost behaviour.
At the start of the level, Blinky and Pinky ghosts will move randomly for the first couple of seconds before falling into their typical behaviour.
This not only adds a small ripple to the gameplay for experienced players but can also serve a good challenge to overcome and enhance the algorithmic complexity of in my solution.
If I have enough time, I will add this.

\item Ms. Pac-Man adds three new mazes that appear back-to-back for each level.
These new mazes have two pairs of warp tunnels.
This allows the player more options to evade the ghosts yet also gives the ghosts more options to intercept the player.
While a nice addition, I will not prioritise it since I would like to get the core gameplay loop down pat before adventuring into other aspects of the solution.

\item The bonus fruit is that appears in the original now moves through the maze.
This makes the eating of it more frantic since you need to pay attention to the fruit is along with the ghosts.
While a neat addition that can add extra challenge for experienced players, I am afraid that this may confuse novices into believing that it is just another threat.
They may subconsciously associate anything that moves besides them to be one, so will put unneeded effort into avoiding it, making the game harder for themselves.
While I will not consider adding this for the meantime. I may add it if I can use it to showcase a programming technique. Maybe the bonus fruit can inherit the movement of the ghosts using OOP?
\end{itemize}

\subsection{Pac-Man Championship Edition (CE)}
\label{sec:org230af21}
\begin{figure}[htbp]
\centering
\includegraphics[width=0.50\linewidth]{./img/analysis/champ/midgame.jpg}
\caption{\label{fig:pacman-cs-midgame}Pac-Man CE midgame.}
\end{figure}

Originally released in 2007, Pac-Man CE is of now the last game that series creator Toru Iwatani has worked on.
This game can be considered Iwatani’s faster-paced re-imagining of Pac-Man with modern game design sensibilities.
When distilled, gameplay follows the same timeless loop as the original, but with some new major twists:

\begin{itemize}
\item Pellets only appear in small clusters at defined points in the maze.
Once all the pellets are eaten a bonus fruit appears that will reveal a new cluster.
While I like this change, I believe that this was done to suit the faster-paced gameplay and would be too large of a change for my recreation of the original and may even alienate stakeholders.

\item Sleeping ghosts can found across the maze, who will wake up and pursue Pac-Man when he moves past them.
This is to incentivise the player to get along trial of ghosts behind them, only to get a power pellet and eat them all in a row.
This adds another loop to gameplay alongside eating pellets which is very satisfying to execute properly.
But like the clusters, I believe that this is too much of a far cry from the original game I am trying to replicate.
The long trail of ghosts can clog up the smaller maze of the original and while it should be trivial to instance all those ghosts using OOP, the sheer number can be too much for Python and Pygame.
Performance issues can be encountered that I may not have enough time to solve.
\end{itemize}

\begin{figure}[htbp]
\centering
\includegraphics[width=0.50\linewidth]{./img/analysis/champ/eating.jpg}
\caption{\label{fig:pacman-eating}The player eating a trail of ghosts behind them.}
\end{figure}

\begin{itemize}
\item Assistive effects are also added.
Game speed slows down when Pac-Man is close to a pursuing ghost.
Pac-Man can also use bombs to send all the ghosts back into their house at the cost of restarting the dot multiplier.
I understand the need for slow motion since this game is significantly faster-paced, I felt like the tension of a ghost almost catching you from behind, only to narrowly escape them by turning a corner is reduced.
I feel like this and the bombs do not gel well with the slower pace of the original.
Though some of my stakeholders likely the children - may want these assistive effects, so I could add them as an option to enable if I have time.
\end{itemize}

\subsection{Pac-Mania}
\label{sec:orgea94dad}
\begin{figure}[htbp]
\centering
\includegraphics[width=0.33\linewidth]{./img/analysis/mania/midgame.jpg}
\caption{\label{fig:mania-mid}Pac-Mania mid level.}
\end{figure}

Released in 1987 for arcades, Pac-Mania is an isometric representation of the original, with some major additions:

\begin{itemize}
\item Pac-Man can now jump over ghosts to serve as another option to evade them.
This adds a tactical ripple to the gameplay.
This makes more sense in an isometric perspective compared to a top-down one.
While still possible top-down with some dynamic sprite scaling it seems like too much effort to implement into the solution for what it is worth.

\item To balance out the jump, two new ghosts are added.
Funky and Splunky.
Both can jump and Pac-Man cannot jump over Splunky.
Like the jumping mechanic, it seems like this would require a lot of attention to implement properly, which would detract from the quality of the core parts of the solution.
The extra ghosts may also cramp up the small maze and confuse the player.

\item Eating bonus fruit will speed up Pac-Man, serving as a power-up in distancing him from the ghosts, but also a power-down since it makes it harder to navigate the maze.
This seems like a nice twist to gameplay, but I will need to see whether this will play nice in the smaller maze of the original.
\end{itemize}

\subsection{What and What not to Add}
\label{sec:orgfdf4a3b}
Here is a table summarising the extra features that I would like and would not like to feature in my solution.
I feel like my choices show that I only want to make conservative additions/changes.

\begin{itemize}
\item \textbf{Will add:}:
\begin{itemize}
\item Solid colour for maze walls.
\end{itemize}

\item \textbf{Tentative (depends on time left and stakeholder feedback):}
\begin{itemize}
\item Multiple maze layouts (two pairs of tunnels).
\item Moving bonus fruit.
\item Initially random ghost movement.
\item Assistive effects like slow-motion and bombs.
\end{itemize}

\item \textbf{Will not add:}
\begin{itemize}
\item Clustered pellets.
\item Sleeping ghosts.
\item More than four ghosts.
\item Jumping for Pac-Man and ghosts.
\end{itemize}
\end{itemize}

\subsection{Features of the Solution}
\label{sec:org173e079}
\begin{itemize}
\item \textbf{The player must be able to control Pac-Man in four cardinal directions.}
\begin{itemize}
\item Source: research.
\item Essential.
\item Pac-Man will move accordingly to the player pressing the movement keys.
\item Moving Pac-Man will allow the player to achieve their goal of eating all the pellets.
\end{itemize}
\end{itemize}


\begin{itemize}
\item \textbf{Maze must be loaded and rendered correctly.}
\begin{itemize}
\item Source: research.
\item Essential.
\item The maze must load in within a reasonable amount of time and render correctly.
\item Gives the player a sense of their surroundings and allows them to correctly navigate the maze.
\end{itemize}
\end{itemize}


\begin{itemize}
\item \textbf{Pellets and power pellets must be rendered within the maze.}
\begin{itemize}
\item Source: research.
\item Essential.
\item Pellets and power pellets must be loaded and rendered alongside the maze.
\item Provides the player with the goal of eating all the pellets.
\end{itemize}
\end{itemize}


\begin{itemize}
\item \textbf{Ghosts must use an algorithm to navigate the maze.}
\begin{itemize}
\item Source: research.
\item Essential.
\item The ghosts will use the same algorithm as the original release.
Each cardinal direction is evaluated and omitted if it leads into a wall or back to where the ghost came.
If the direction passes, the distance between the next tile and target tile is calculated.
The direction with the shortest distance will be picked.
\item Gives the player the challenge of avoiding the ghosts.
\end{itemize}
\end{itemize}


\begin{itemize}
\item \textbf{Each ghost must replicate their unique behaviour from the original.}
\begin{itemize}
\item Source: research.
\item Essential.
\item Each ghost will target a different tile to navigate to.
During chases mode, Blinky and Clyde’s tile will be on Pac-Man, Pinky two tiles ahead of Pac-Man and Inky a tile calculated using Pinky’s tile and Blinky’s position.
\item More challenge for the player since they must memorise the behaviour of each ghost.
\end{itemize}
\end{itemize}


\begin{itemize}
\item \textbf{Ghosts must switch between scatter and chase modes according to a timer.}
\begin{itemize}
\item Source: research.
\item Essential.
\item Each ghost will monitor a timer and switching between chasing Pac-Man and fleeing to the corners of the maze.
\item Scatter mode will allow the player to take a short break from the pursuit.
\end{itemize}
\end{itemize}


\begin{itemize}
\item \textbf{Ghosts must become temporarily frightened when Pac-Man eats a power pellet.}
\begin{itemize}
\item Source: research.
\item Essential.
\item The ghosts will turn blue, turn around and move in random directions for several seconds.
\item A pivotal part of the gameplay loop; it allows the player to eat them.
\end{itemize}
\end{itemize}


\begin{itemize}
\item \textbf{Pac-Man must be able to eat frightened ghosts.}
\begin{itemize}
\item Source: research.
\item Essential.
\item Once eaten, a ghost will become a pair of eyes and flee back to the ghost house to regenerate.
The player is rewarded with bonus points.
\item Completes the gameplay loop of Pac-Man being hunted by the ghosts and turning the tables to hunt them instead.
\end{itemize}
\end{itemize}


\begin{itemize}
\item \textbf{The level must be considered complete when Pac-Man eats all the pellets.}
\begin{itemize}
\item Source: research.
\item Essential.
\item After eating all the pellets, the ghosts will disappear, and Pac-Man will remain still for several seconds while the maze flashes white.
The game will then proceed to the next level.
\item Tells the player that they passed the level.
\end{itemize}
\end{itemize}


\begin{itemize}
\item \textbf{The bonus fruit must appear to be eaten.}
\begin{itemize}
\item Source: research.
\item Essential.
\item A fruit will appear in a fixed location in the maze after some time has elapsed in the maze.
Pac-Man can eat this to get bonus points.
\item Gives the player a small extra challenge that shouldn’t take long to program in.
\end{itemize}
\end{itemize}


\begin{itemize}
\item \textbf{Pac-Man must die when he gets caught by an un-frightened ghost.}
\begin{itemize}
\item Source: research.
\item Essential.
\item When caught, Pac-Man will remain frozen in place and playout his death animation and loses a life.
If he still has lived the player would resume the level, else they’ll see a game over screen.
\item Adds stakes to the game.
The player will be incentivised to not make too many mistakes.
\end{itemize}
\end{itemize}


\begin{itemize}
\item \textbf{The time spent between scatter and chase modes decreases over time.}
\begin{itemize}
\item Source: research.
\item Essential.
\item After being in scatter mode for three times, the next scatter will only last 5 seconds instead of 7.
\item Adds a difficulty curve to the game.
\end{itemize}
\end{itemize}


\begin{itemize}
\item \textbf{Blinky will always be in chase mode when there is only a few pellets left in the maze.}
\begin{itemize}
\item Source: research.
\item Essential.
\item Blinky will always target Pac-Man when there is 20 or less pellets left.
\item Adds a difficulty curve to the game.
\end{itemize}
\end{itemize}


\begin{itemize}
\item \textbf{Ghost behaviour will become more aggressive the more levels the player completes.}
\begin{itemize}
\item Source: research.
\item Essential.
\item The initial durations between scatter and case mode will become smaller the more levels are completed.
\item Adds a difficulty curve to the game.
\end{itemize}
\end{itemize}


\begin{itemize}
\item \textbf{Blinky and Pinky’s random movement.}
\begin{itemize}
\item Source: research.
\item Essential.
\item Blinky and Pinky will behave like they’re in frighted mode for the first couple of seconds of a level.
\item Adds a ripple to gameplay that can catch out experienced players.
It will also add to the algorithmic complexity of the ghosts.
\end{itemize}
\end{itemize}


\begin{itemize}
\item \textbf{Replicate ‘up’ unit vector overflow error.}
\begin{itemize}
\item Source: research.
\item Not essential.
\item In the original release.
The code that handles adding magnitude to unit vectors. treats the x and y components as one 16-bit value rather than two 8-bit ones, leading to the multiplication in the y component overflowing into the x.
\item I decided to reject this since it’s obviously a bug, likely unintended by the developers.
\end{itemize}

\item \textbf{Title Screen}
\begin{itemize}
\item Source: own idea.
\item Essential.
\item Displays the title of the game along with a prompt asking the user to press a key to begin the game.
\item Allows the player time to prepare before playing.
\end{itemize}

\item \textbf{Pause Screen}
\begin{itemize}
\item Source: own idea, stakeholders.
\item Essential.
\item Pause and resume execution of the game on demand using a key stroke.
\item Allows the player to readjust their posture and or have a break.
\end{itemize}

\item \textbf{Pause screen menu.}
\begin{itemize}
\item Source: own idea.
\item Not essential.
\item A menu appears upon pausing the game, asking the player if they want to resume or quit.
\item While a nice feature in terms of player accessibility.
This functionality can be replicated using key strokes, therefore it’s not essential.
\end{itemize}

\item \textbf{Solidly coloured maze walls.}
\begin{itemize}
\item Source: research, stakeholders.
\item Essential.
\item Maze walls will consist of a solid colour instead of outlines.
\item Makes it easier for the player to see the maze, especially if the player is visually impaired and or is playing on a small display.
\end{itemize}

\item \textbf{High score and leader board system.}.
\begin{itemize}
\item Source: stakeholders.
\item Not essential.
\item High scores will be saved and added to a leader board to compare.
\item Enables a sense of competition among players.
I decided to omit this feature in the interest of time.
\end{itemize}

\item \textbf{Multiple mazes.}
\begin{itemize}
\item Source: research stakeholders.
\item Not essential.
\item Each level will have one of four mazes, each with a unique layout.
\item Provides come variety to the game.
\item Will be omitted or added later in the interest of time.
\end{itemize}

\item \textbf{Bouncing bonus fruit.}
\begin{itemize}
\item Source: research.
\item Not essential.
\item The bonus fruit that appears will move around the maze like a ghost.
\item Adds a ripple to by making the player pay attention to both the fruit and ghosts.
\item Decided to omit this feature due to the aforementioned confusion it can cause in novice players.
\end{itemize}
\end{itemize}

\section{Limitations}
\label{sec:org8d89cf5}
This project will not end up being a feature rich or ‘complete’ remake of Pac-Man.
I want to make an expressed focus on implementing the core mechanics of the original game and extra features afterwards.
These extra features will be simple to implement, like solidly coloured walls, as to not distract from the progress of implementing the core features.
Pac-Man’s movement, the ghosts’ movement, win and lose conditions and behaviour states are deemed the most essential and will receive the most attention.
Features like bouncing bonus fruit, multiple mazes and a high score system will be either omitted outright or have a chance of being implemented later.
My omission of curtain features is done so that I know I have enough time to finish implanting the core features before the deadline.
One example is a pause menu.
Instead of the player clinking on a button to resume the game, they can instead press a key like escape to achieve the same thing.
This saves me from having to program a pause menu and figure out its design, allowing me to focus on the core features.

\section{Requirements for the Solution}
\label{sec:org5ac2d6e}
The project will be developed on a Microsoft Windows machine, but will have multi-platform compatibility in mind.
While Python and Pygame are multi-platform, I still need to look out for subtle differences.
One example is using backward slashes for directories in my code despite forward slashes being used in *nix systems.
Pythons built in module ‘os’ will solve this discrepancy.

For this project to succeed, I would need to:

\begin{itemize}
\item Have access to a computer running an operating system compatible with Python and Pygame.
\item Use the programming language Python with modules from its standard library and any external libraries when necessary.
\item Use the game library Pygame and all its relevant functions and methods.
\item Use an image manipulation program to edit sprites and make concepts.
\end{itemize}

\subsection{System Requirements}
\label{sec:orgc018c4d}
\begin{itemize}
\item OpenGL API for graphics rendering.
\item OpenAL API for audio.
\item MinGW 4.8.1 or Microsoft Visual C++ Express 2010 (MSVC).
\item OpenAL Windows driver.
\item Operating Systems: Linux, Windows, MacOS, Free BSD, Open BSD \autocite{pygameAbout} \autocite{SDLIntro}.
\end{itemize}

Failure to meet these requirements will lead to the development of the game being delayed or in the worst case, unable to continue.
In terms of the end user, their hardware and or software may not be powerful enough to run the game or is missing essential requirements for Pygame to operate correctly, such as OpenGL.
But most devices today should meet these relatively low system requirements.

\subsection{Success Criteria}
\label{sec:orgd5a32bd}
\begin{itemize}
\item \textbf{Make the game boot up and display a window.}
\begin{itemize}
\item Category: functionality.
\item Measurement: development testing.
\item This is the only means for the player to access and interact with the game.
\end{itemize}
\end{itemize}


\begin{itemize}
\item \textbf{Have the game be able to display Pac-man and be capable of switching between animation frames.}
\begin{itemize}
\item Category: functionality.
\item Measurement: development testing.
\item Pac-man will be the object that the player controls, it be the most focused on aspect in the game.
It is important that Pac-man looks good.
\end{itemize}
\end{itemize}


\begin{itemize}
\item \textbf{The player can press another movement key while holding another and Pac-Man will update his movement to the newly pressed key.}
\begin{itemize}
\item Category: Usability, robustness.
\item Measurement: development testing.
\item Pac-man’s movement must not only feel responsive, but also initiative for new and experienced players.
A new player may need time to get a grasp of the controls, which means their fingers may get muddled.
Pac-Man responding to the last key pressed ensures that he promptly responds to player intentions.
\end{itemize}
\end{itemize}


\begin{itemize}
\item \textbf{Have Pac-Man be able to move in all four directions while keeping aligned to the grid of tiles.}
\begin{itemize}
\item Category: functionality, robustness.
\item Measurement: development testing.
\item Allows the player to make Pac-man fully traverse the maze and avoid the ghosts, keeping Pac-man fixed to the tiles will prevent Pac-man from clipping out of the maze.
\end{itemize}
\end{itemize}


\begin{itemize}
\item \textbf{Allow Pac-Man to move smoothly between the tiles.}
\begin{itemize}
\item Category: usuability, cosmetic.
\item Measurement: development testing.
\item The smooth movement will give the player the enough reaction time to decide whether they should turn a corner or continue to move forward.
It also ensures that Pac-man looks good and feels good to move.
\end{itemize}
\end{itemize}

\printbibliography
\end{document}
